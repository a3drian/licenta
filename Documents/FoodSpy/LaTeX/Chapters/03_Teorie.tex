% !TeX root = ../FoodSpy.tex
% \section{Aplicații web - noțiuni teoretice}

\section{Aplicație web vs. site web}
Deși se accesează în același fel și arată la fel, o aplicație web nu este un website.\\
O aplicație web este construită cu scopul de a interacționa cu utilizatorul, în timp ce un website doar servește conținut static, conținut care nu poate fi afectat de către vizitator.
\\ \\
Website-ul este structurat pe mai multe pagini care au adrese web diferite și care indică resurse web diferite, în timp ce, de regulă, o aplicație web ”rescrie” în mod dinamic pagina curentă pe care se află utilizatorul; această ”rescriere” stă la baza așa numitelor ”single-page application”.
\\ \\
Pentru a putea interacționa cu aplicația web, un utilizator trebuie autentificat și autorizat \footnote{Dacă luăm un exemplu cu un club, prin autentificare se înțelege procesul de identificare a vârstei unui petrecăreț, pe baza unui buletin, în vreme ce prin autorizare se înțelege acordarea unor privilegii speciale acestui petrecăreț, accesul în zona VIP, în zona DJ-ului, și așa mai departe.}, în vreme ce vizitatorul unui website poate cel mult să se înscrie pe o listă de corespondență de unde poate primi notificări din partea administratorul website-ului.


\section{Arhitectura client-server}


\section{HTTP și REST}

